\documentclass[a4paper]{article}

\usepackage[swedish]{babel}
\usepackage[T1]{fontenc}
\usepackage[utf8]{inputenc}
\usepackage{courier}
\usepackage{listings}
\usepackage{pdfpages}

\title{Minneshantering}
\date{ID2200 - Laboration 3}
\author{Cecilia Roes \\ Fredrik Hillnertz}

\begin{document}
\includepdf{pre_lab3.pdf}
\maketitle

\section*{Problembeskrivning}

\section*{Programbeskrivning}
% Skriv om metod för hopslagning av minnesblock (2.2 punkt 6)
% Redan impl i free.

\subsection*{Förklaring av kod som kompileras om MMAP är definierat}

\subsubsection*{I koden initialiseras \texttt{\_\_endHeap} mha \texttt{sbrk(2)} vilket inte är så snyggt och inte skulle fungera på system där \texttt{sbrk(2)} inte stöds. Vad händer om du tar bort dessa initialiseringar (två \texttt{if}-satser)? Vad skulle \texttt{\_\_endHeap} då representera?}

\subsubsection*{Vad händer om man skickar \texttt{NULL} som första parameter till \texttt{mmap(2)}? Vad fungerar/fungerar inte? Vet man då var minnet reserveras?}

\subsubsection*{Vad händer om man byter ut/tar bort flaggan \texttt{MAP\_SHARED} i anropet till \texttt{mmap(2)}?}

\section*{Utvärdering}
% Lägg till -O4 till kompileringen för maximal optimering

% Liten, stor och blandad storlek på data, mycket och lite data (en/flera allokeringar).
% Repetera körningarna, använd någon form av snittvärden.

\subsection*{First fit}

\subsection*{Best fit}

\section*{Filkatalog}
Alla filer som är associerade med denna lab går att finna i:\\
\texttt{frhi@shell.it.kth.se:/Kurser/OS/lab3/}
\\
Filer:
\begin{description}
\item[malloc.c] källkoden
\item[Makefile] makefil för att bygga programmet
\item[lab3-rapport.pdf] denna rapport i pdf-format
%\item[exampleRuns.txt] exempel på körningar av programmet
\end{description}

\section*{Utskrift med kompileringskommandon}
%\lstinputlisting[basicstyle=\footnotesize, breaklines]{commands.txt}

\section*{Källkod}
%\lstinputlisting[language=C, xleftmargin=-1in, basicstyle=\footnotesize]{smallshell.c}

\section*{Verksamhetsberättelse och synpunkter}
Tidsåtgång: ca h per person.

\end{document}

