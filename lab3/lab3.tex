\documentclass[a4paper]{article}

\usepackage[swedish]{babel}
\usepackage[T1]{fontenc}
\usepackage[utf8]{inputenc}
\usepackage{courier}
\usepackage{graphicx}
\usepackage[justification=centering]{caption}
\usepackage{listings}
\usepackage{color}
\usepackage{pdfpages}
\definecolor{mygray}{rgb}{0.5,0.5,0.5}
\lstset{language=C, basicstyle=\footnotesize,
        numbers=left, numberstyle=\tiny\color{mygray}}

\title{Minneshantering}
\date{ID2200 - Laboration 3}
\author{Cecilia Roes \\ Fredrik Hillnertz}

\begin{document}
\includepdf{pre_lab3.pdf}
\maketitle

\section*{Problembeskrivning}
Implementera de metoder som förvaltar systemets minnesresurser nämligen: \texttt{malloc}, \texttt{free} och \texttt{realloc}. I uppgiften ingick också att implementera ett antal olika taktiker för minnesallokering i \texttt{malloc}, såsom first-fit, best-fit och worst-fit. Kraven på implementation är att den skall följa \texttt{ANSI/ISO}-standarden och att inga metoder ska läcka minne.
\\
\\
Utöver detta så ska prestandan evalueras. I testerna ska worst och best case scenarion ingå, samt rimliga användningsfall. Med hjälp av testerna så ska minnesallokeringens prestanda med avseende på minnesutnyttjande och exekveringstid mätas, utvärderas och jämföras med prestandan för biblioteksimplementationen av \texttt{malloc}.

\section*{Programbeskrivning}
Grunden för programmet är pekare till en cirkulär länkad lista med oanvända minnesblock, listan innehåller \texttt{Headers}. En \texttt{Header} är en union av en struct och en aligment variabel för att tvinga aligment av blocken. Structen består av en pekare till nästa \texttt{Header} i listan och en variabel för det lediga blockets storlek.
\\
Programet har fyra viktiga funktioner som modifierar den länkande listan på olika sätt: \texttt{malloc}, \texttt{free}, \texttt{morecore} och \texttt{realloc}. För att förstå varje enskild funktion krävs det att man också förstår vad de andra gör då de är tätt sammankopplade. 
\\\\
\texttt{free}: \texttt{free} är till för att lämna tillbaka ett block av minne så att det kan återanvändas. Funktionen tar en pekare till minnesblocket som ska återlämnas. \texttt{free} går igenom den länkade listan och hittar hålet där blocket passar in baserat på minnesaddressen. Blocket sätts in i listan genom att ändra på pekarna till block för blocket innan och det som sätts in. \texttt{free} slår också ihop minnesblock till ett om dom ligger efter varandra i minnet. Detta beskrivs i mer detalj senare. \texttt{free} returnerar inget utan sätter istället pekaren till länkade listan till det frigjorda blocket.
\\\\
\texttt{morecore}: \texttt{morecore} anropas av \texttt{malloc} när det inte finns tillräckligt med minne i den länkade listan med lediga minnesblock. Som parameter tar \texttt{morecore} antalet `units` som behöves. En unit motsvarar storleken på en \texttt{Header}. På så sätt kommer allt vara alignat i minnet vilket gör det mer effektivt. \texttt{morecore} ber systemet om mer minne genom att kalla på \texttt{mmap} eller \texttt{sbrk} (mer detaljer om detta senare). Av systemet får man förhoppningsvis en pekare till början av det allokerade minnet annars så printas ett errormeddelande och \texttt{NULL} returneras. Om allt går väl så skapas en ny \texttt{Header} som pekar på där det nyallokerade minnet börjar. Storleken för \texttt{Headern} sätts till det antal bytes som allokerades. Pekaren till den nya \texttt{Headern} skickas sedan till \texttt{free} som sätter in blocket i den länkade listan. Sedan returneras pekaren till den länkade listan som \texttt{free} har uppdaterat till det nya blocket.
\\\\
\texttt{malloc}: \texttt{malloc} är till för att allokera minne. Funktionen tar en parameter: antalet bytes som efterfrågas. Det första som räknas ut är antalet units som behövs, en unit för en \texttt{Header} plus antalet units som krävs för att rymma de efterfrågade bytesen. Om det är det första anropet till \texttt{malloc} så initieras den länkade listan med ett tomt block. Sedan börjar \texttt{malloc} leta efter ett passande block i den länkade listan. Detta kan ske på tre olika sätt beroende vilken strategiflagga som har satts vid kompileringen. Strategierna är: \texttt{First fit}=1, \texttt{Best fit}=2 och \texttt{Worst fit}=3. 
\\\\
Skillnaden mellan strategierna är hur blocken i listan väljs. \texttt{First fit} avslutar så snabbt ett tillräckligt stort block har hittats. \texttt{Best fit} går igenom hela listan och tar det block som passar bäst om inte ett block som passar perfekt hittas. Då väljs det direkt. \texttt{Worst fit} går också igenom hela lista men väljer det block som passar sämst. Alltså det största blocket som också är stort nog. Metoderna har dock en del gemensamt. 
\\
Varje metod kör en for-loop som börjar vid pekaren till den länkade listan. Vid slutet av varje iteration följs pekaren till nästa \texttt{Header}. I varje iteration undersöks storleken på det block som hanteras. Om det är tillräckligt stort så kollas det antingen om blocket är det bästa hittills eller så väljs det direkt. Den länkade listan är som sagt cirkulär. Om man har gjort ett helt varv och är tillbaka till den \texttt{Header} man började på och man ännu inte har hittat ett passande block innebär det att det inte finns ett tillräckligt stort block i \texttt{free}-listan. Då anropas \texttt{morecore} med det antal units som behövs. Om \texttt{NULL} returneras från \texttt{morecore} innebär det att det inte finns mer minne, då returneras \texttt{NULL}. Om inte så finns det nu ett extra block i \texttt{free}-listan som är tillräckligt stort. Detta block hittas tillslut då algoritmerna fortsätter att gå igenom listan.
\\
När ett passande block har hittats så är det antingen större än tillräckligt eller så passar det perfekt. Om det passar perfekt tas blocket ut ur \texttt{free}-listan och returneras. Om inte så trimmas blocket. Detta görs genom att först dra av antalet units som behövs från \texttt{Headerns} storleksvariabel. Nu finns det ett "hål" i minnet. Pekaren flyttas fram till hålet genom att addera storleken på det reducerade blocket. Pekaren flyttas alltså fram ett visst antal \texttt{size*sizeof(Header)} steg i minnet. Detta blir en ny \texttt{Header} vars storlek sätts till det antal units som skulle allokeras från början och returneras. Trimningen har effekten att den delar upp det valda blocket i två delar, en del som passar perfekt och en del med det minne som blir över. Den sistnämnda får sin storlek reducerad och blir kvar i \texttt{free}-listan.
\\
När blocket returneras vill man inte returnera pekaren till \texttt{Headern}, för då tillåter man den anropande metoden att skriva över den administrativa informationen. Det som returneras är pekaren till \texttt{Headern} + 1, adressen till det lediga blocket minne man just allokerat.
\\\\
\texttt{realloc}: \texttt{realloc} är till för att ändra på storleken för minne som redan har allokerats. Den tar en pekare till minnet som ska reallokeras och antalet byte som man vill att minnet ska ändras till. Om pekaren till minnet är \texttt{NULL} fungerar \texttt{realloc} precis som \texttt{malloc}. Annars räknas antalet units som kommer behövas ut. Den gamla storleken hämtas genom att subtrahera ett från pekaren så att man får ut administrations-\texttt{Headern} som har en storleks-variabel. Sedan anropas \texttt{malloc} för att allokera det nya området. Om det inte finns mera minne returneras den gamla pekaren och ingenting ändras. Om det finns mera minne tillgängligt så kopieras all den gamla datan som pekaren pekar ut till det minne som allokerades av \texttt{malloc}. Kopieringen görs med \texttt{memcpy}. Självklart så kopieras inte all data om det nya området är mindre än det gamla. Tillsist frigörs den gamla pekaren med \texttt{free} och den nyallokerade pekaren returneras. 


\subsection*{Hopslagning av lediga minnesblock }
Det mest logiska är att slå ihop minnesblock i metoden free. När free anropas så hittas hålet i free-listan där blocket passar in. Sedan undersöks det om det föregående och/eller det efterföljande blocket i listan ligger brevid varandra i minnet. Isåfall så slås de kontinuerliga blocken ihop till ett. Detta innebär att alla block är så ihopslagna som möjligt när malloc anropas. Free-listan är alltså så kort som det bara går vilket gör malloc mer effektiv då det går snabbare att gå igenom en liten lista.

\subsection*{Förklaring av kod som kompileras om MMAP är definierat}
\lstinputlisting[firstline=66, lastline=74, firstnumber=66]{malloc.c}
Variabeln \texttt{\_\_endHeap} är en pekare som pekar på slutet av processens minne. \texttt{\_\_endHeap} iniatiliseras med värdet av anropet \texttt{sbrk(0)} som returnerar en pekare till slutet av processens minne.

\lstinputlisting[firstline=77, lastline=92, firstnumber=77, breaklines=true]{malloc.c}
I funktionen \texttt{morecore} ber man systemet om mer minne. Om \texttt{MMAP} är definerat gör detta med ett anrop till \texttt{mmap} som skapar en ny mappning i det virituella minnet för den anropande processen. \texttt{mmap} tar sex stycken parametrar. 1: En startaddress där minnet ska börja mappas. Här skickas \texttt{\_\_endHeap} in så att minnet blir kontinuerligt. 2: Hur mycket minne som ska mappas i bytes. Detta värde är en multipel av systemets page size som hittas med \texttt{getpagesize()}. 3: Flaggor som sätter minnesskyddet för det mappade minnet. Flaggorna som vi använder är \texttt{PROT\_READ} och \texttt{PROT\_WRITE} så att minnet kan skrivas och läsas. 4: Mer flaggor. De som används är \texttt{MAP\_SHARED} vilket innebär att mappningen är synlig för andra processer, och \texttt{MAP\_ANONYMOUS} som gör att mappningen inte "backas up" av en fil, minnet som allokeras initieras till noll och de två resterande argumenten ignoreras. Efter att \texttt{mmap} har körts flyttas \texttt{\_\_endHeap} fram med lika många bytes som allokerades.

På rad 88 beräknas det antal sidor som kommer allokeras, värt att notera är att antalet är beroende av den begärda minnesstorleken (dock minst \texttt{NALLOC}) och storleken av \texttt{Header}.
\\
\subsubsection*{I koden initialiseras \texttt{\_\_endHeap} mha \texttt{sbrk(2)} vilket inte är så snyggt och inte skulle fungera på system där \texttt{sbrk(2)} inte stöds. Vad händer om du tar bort dessa initialiseringar (två \texttt{if}-satser)? Vad skulle \texttt{\_\_endHeap} då representera?}
Vid första anropet kommer \texttt{\_\_endHeap} att peka på \texttt{NULL} vilket gör att systemet väljer vart minnet ska mappas när \texttt{mmap} anropas (se nästa fråga). Vid efterföljande anrop till \texttt{mmap} kommer \texttt{\_\_endHeap} att peka ut en address. Systemet tar dock bara addresspekaren som ett förslag. Om det inte går att mappa in minne där så väljer systemet än en gång vart minnet ska mappas. Ibland hamnar mappningen där \texttt{\_\_endHeap} pekar, ibland inte. Resultatet blir att minnet blir segmenterat. \texttt{\_\_endHeap} visar nu inte längre vart i det virtuella minnet programmet är utan den blir en slags räknare som representerar hur mycket minnet som har mappats med \texttt{mmap}.

\subsubsection*{Vad händer om man skickar \texttt{NULL} som första parameter till \texttt{mmap(2)}? Vad fungerar/fungerar inte? Vet man då var minnet reserveras?}
Då bestämmer kärnan vart minnet ska mappas in. När vi har testat har alla test gått igenom. Risken finns dock att \texttt{malloc} blir mindre effektivt om kärnan allokerar minnet väldigt utspritt. Då kommer olika block inte kunna mergas ihop vilket leder till att free-listan blir längre, och det gör att det tar längre tid att gå igenom den.

\subsubsection*{Vad händer om man byter ut/tar bort flaggan \texttt{MAP\_SHARED} i anropet till \texttt{mmap(2)}?}
Flaggorna \texttt{MAP\_SHARED} och \texttt{MAP\_PRIVATE} styr huruvida uppdateringar av mappningen är synlig för andra processer eller inte, det avgör också om uppdateringar görs på den underliggande filen. En av flaggorna måste specificeras. Byter man till \texttt{MAP\_PRIVATE} kommer en privat copy-on-write mappning skapas, uppdateringar syns inte för andra processer och skrivs inte till fil.

\section*{Utvärdering}
\subsection*{Analys}
Nedan följer kommentarer och slutsatser från att ha studerat källkoden. Dessa stämmer inte nödvändigtvis med programmets faktiska prestationer.

\subsubsection*{Minnesåtgång}
Alla de implementerade allokeringsmetoderna har samma bästa och värsta fall, sedan är de olika bra på att hantera andra allokeringsbeteenden/mönster.
\\
I bästa fall allokeras precis så mycket minne som får plats i det lediga blocket (antingen genom ett eller flera anrop, vid flera anrop kommer bakre änden successivt allokeras).
\\
I värsta fall allokeras så mycket minne att påföljande minnesallokeringar precis inte får plats, detta sker då man allokerar hälften av ett tillgängligt block (eftersom headern tar upp en bit minne utöver det man vill allokera så får nästa allokering inte plats).

\subsubsection*{Tidsåtgång}
Tidskomplexiteten i värsta fall är samma oavsett allokeringsmetod, om inget tillräckligt stort minne hittas gås hela listan igenom. Däremot kommer best- och worst-fit gå igenom hela listan varje gång (undantaget en liten optimering av best-fit som gör att den slutar vid perfect-fit) medan first-fit slutar då ett tillräckligt stort block hittats.

Bästa fallet för tidsåtgång sammanfaller med bästa fallet för minnesåtgång. Om man helt fyller upp alla minnesblock så kommer listan bara innehålla 1 element, och att leta blir en konstant operation. Däremot måste man köra \texttt{mmap}, vilket är en tung operation, varje gång ett block fylls upp, så att fylla det med små bitar är det bästa.

Även värsta fallet är samma som för minnesåtgång. När blocken aldrig fylls upp innebär det att listan växer för varje ny allokering, listan får alltså samma längd som antalet allokeringar.

\begin{table}[h]
\centering
\caption{Resursåtgång i värsta och bästa fall \\ m = summa av begärt minne, n = antal malloc-anrop}
\scalebox{1.2}{
  \begin{tabular}{|l|c|c|}
    \hline
    & \multicolumn{2}{c|}{Åtgång} \\
    \hline
    Fall & Minne & Tid \\
    \hline
    Bästa & m + headers & 1 \\
    Värsta & 2*m & n \\
    \hline
  \end{tabular}
}
\end{table}

\subsection*{Evaluering}
Som diskuterats ovan har vi alltså ett tydligt bästa och värsta fall, och eftersom vi har tillgång till källkoden vet vi vad lämpliga parametrar till malloc är för att undersöka dessa.

I testet för värsta fallet anropar vi \texttt{malloc(512*sizeof(Header))}, vilket är halva storleken av det minsta block som skapas av \texttt{morecore}.

I testet för bästa fallet anropar vi \texttt{malloc(1023*sizeof(Header))}, vilket är storleken av det minsta block som skapas av \texttt{morecore} minus en header. Detta innebär att varje anrop av \texttt{malloc} kommer fylla upp ett nytt block som allokerats av \texttt{morecore}.

Vi har dessutom test för ökande respektive minskande minnesstorlekar och ett test med randomiserade minnesstorlekar för att bättre efterlikna verkligt användande.

Evalueringen är körd på en Dell Vostro 3450 under Ubuntu 12.04 64-bit.
Scriptet kördes 10 gånger och vad som visas i tabellerna är medelvärden av dessa körningar.

\begin{table}[h]
\centering
\caption{Tidsåtgång för de olika implementationerna}
\scalebox{1.0}{
  \begin{tabular}{|l|c|c|c|c|c|}
    \hline
    & \multicolumn{5}{c|}{Testfall} \\
    \hline
    Implementation & Bästa & Värsta & Random & Ökande & Minskande \\
    \hline
    First fit & & & & & \\
    Best fit & & & & & \\
    Worst fit & & & & & \\
    Referens & & & & & \\
    \hline
  \end{tabular}
}
\end{table}

\begin{table}[h]
\centering
\caption{Minnesåtgång för de olika implementationerna}
\scalebox{1.0}{
  \begin{tabular}{|l|c|c|c|c|c|}
    \hline
    & \multicolumn{5}{c|}{Testfall} \\
    \hline
    Implementation & Bästa & Värsta & Random & Ökande & Minskande \\
    \hline
    First fit & & & & & \\
    Best fit & & & & & \\
    Worst fit & & & & & \\
    Referens & & & & & \\
    \hline
  \end{tabular}
}
\end{table}

\subsection*{Slutsats}
Gör man multipla körningar i samma process (t.ex. genom att loopa anropet av ett testfall flera gånger) kommer minnet som man avallokerat genom anrop till \texttt{free} finnas tillgängligt i listan, så körningar efter den första går väldigt snabbt då inget mer minne behöver allokeras utan allt får plats. Detta beror på att minne som tilldelats processen genom \texttt{MMAP} först släps när processen terminerar.
Detta gör det svårt att få något medelvärde på körningstid eller minnesanvändning, vilket i sin tur gör det svårt att uppskatta felmarginaler. Vi valde att köra vårat evalueringsscript 10 gånger för att få någon uppfattning om dessa, fler körningar hade givetvis varit att föredra men med våra förutsättningar inte praktiskt genomförbart.

\section*{Filkatalog}
Alla filer som är associerade med denna lab går att finna i:\\
\texttt{frhi@shell.it.kth.se:/Kurser/OS/lab3/}
\\
Filer:
\begin{description}
\item[malloc.c] källkoden
\item[eval.c] kod för prestanda evalueringen
\item[RUN\_EVAL] script för att kompilera och köra evalueringen
\item[Makefile] makefil för att bygga programmet
\item[lab3-rapport.pdf] denna rapport i pdf-format
%\item[exampleRuns.txt] exempel på körningar av programmet
\end{description}

\section*{Utskrift med kompileringskommandon}
\lstinputlisting[basicstyle=\footnotesize, breaklines]{commands.txt}

\section*{Källkod}
\subsection*{Ur malloc.c}
\lstinputlisting[xleftmargin=-0.5in, firstline=144, lastline=236, firstnumber=144]{malloc.c}

\subsection*{eval.c}
\lstinputlisting[xleftmargin=-0.5in]{eval.c}

\section*{Verksamhetsberättelse och synpunkter}
Tidsåtgång: ca h per person.

\end{document}

